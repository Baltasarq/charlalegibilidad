\documentclass{article}

\title{Legibilidad del c\'odigo fuente: facilitando el mantenimiento de las
aplicaciones}
\author{Garc\'{\i}a Perez-Schofield J. Baltasar.
   \and Escuela Superior de Ingenier\'ia Inform\'atica
   \and Universidad de Vigo, Espa\~na(Spain).
   \and jbgarcia@uvigo.es
}



\begin{document}
\maketitle

\section{Legibilidad}
En la exposici\'on se proponen diferentes t\'ecnicas, que, asociadas a la
indentaci\'on, pueden
ayudar a escribir c\'odigo m\'as legible. En concreto, se enumeran t\'ecnicas de
escritura de comentarios,
t\'ecnicas de formato de definici\'on de variables, t\'ecnicas de divisi\'on del
c\'odigo en l\'ineas, y
finalmente, escritura correcta de condiciones y bucles.

Adem\'as, se explican tambi�n otras t\'ecnicas, no sint\'acticas, sino
sem\'anticas, que facilitan enormemente la escritura de c\'odigo. En concreto,
se repasa el nombrado de variables, que permite, realiz\'adondolo de manera
exhaustiva, distinguir la funci\'on de la variable por su nombre. El nombrado
de funciones y procedimientos, que debe ser significativo y explicar de manera
clara y concreta la funcionalidad ... etc.

Finalmente, se propone un ejemplo real, sacado de un libro de texto de primero
de carrera, sobre el juego del Nim, con lo que se busca fomentar el esp\'iritu
cr\'itico de los asistentes a la charla.


La \'ultima parte de la exposici\'on, ya de una manera l\'udica como ejemplo de
lo que no se debe hacer, se dedica al
concurso de C-ofuscado, que trata precisamente de programas que parten de
c\'odigos fuente
ilegibles.

\section{Bibliograf\'ia}

\begin{itemize}
\item Eckel (2000). "Thinking in C++". Prentice Hall
\item Caro, Ramos, Barcel\'o (2002). "Introducci\'on a la programaci\'on con
orientaci\'on a objetos". Prentice-Hall
\end{itemize}


\section{Referencias WWW}

\begin{itemize}
\item Documentaci\'on de Microsoft:
\begin{verbatim}
http://msdn.microsoft.com/library/default.asp?

url=/library/enus/

stg/stg/coding_style_conventions.asp
\end{verbatim}
\item Documentaci\'on de Sun:
\begin{verbatim}
http://java.sun.com/docs/codeconv/html/CodeConvTOC.doc.html
\end{verbatim}
\end{itemize}

\end{document}
